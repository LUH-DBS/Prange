\chapter{Proposed Method}\label{chap:proposed_method}
% \section{Overview}\label{sec:overview}
%! A Column containing a NaN or None is not 'unqiue' (because it can not be a primary key)
In this thesis I present a method to increase the efficiency of finding unique columns in a table. The method is based on a machine learning model which uses the first few rows to guess if each column will have any duplicate values. Each positive guess will subsequently be validated using a conventional naive method.

The proposed method works in three steps. First, the features are extracted from the first rows of the table. After that, the model tries to predict the existence of duplicate values from the features. Finally, the columns which are unique according to the model are checked with a naive method.

\section{Extracted Features}\label{sec:extracted_features} % TODO: Writing
The feature extraction is necessary as explained in Section~\ref{sec:machine_learning}

Explain the feature extraction with the function in Listing~\ref{lst:proposed_method-prepare_column}.

An example of a table with the extracted features is the Table~\ref{table:feature_table_example}.

\lstinputlisting[float, language=Python, caption={[Preparing a column]This code shows how a column is prepared for the model. This process is repeated for each row; the result forms the feature table. The variable \texttt{column} contains the first \textit{n} rows of the column where \textit{n} is the input size of the model.}, label={lst:proposed_method-prepare_column}]{table-code/proposed-method/prepare-columns.py}


\section{Training the Model}\label{sec:traing_the_model} % TODO: Writing
How was the model trained? What settings where used for the training and why?
