\section{Correctness}\label{sec:correctness}
The correctness of the model is probably the most important metric to determine its usability. While a false positive is not a major problem because each positive guess is verified (see Chapter~\ref{chap:proposed_method}), a false negative will mean that a primary key candidate gets ignored.

In this section, different experiments will be conducted to determine which parameters are the best to train the model. Additionally, in Section~\ref{subsec:correctness_examine-false-guesses} the column which led to false guesses by the model will be examined.


\subsection{Experiment Data}\label{subsec:correctness_experiment-data} % TODO: Writing
The experiments where performed on the gittables dataset\cite{gittables-article}. It contains around one million tables taken from %TODO: describe the dataset

For these experiments, only the tables with at least 100 rows and 3 columns where used. % TODO: how many exist in the dataset?


\subsection{Comparing models with different input sizes}\label{subsec:correctness_comparing-input-size} % TODO: Writing
Experiment Setup

Result

Conclusion

\subsection{Altering the training time}\label{subsec:correctness_comparing-training-time} % TODO: Writing
Experiment Setup

Result

Conclusion

\subsection{Testing only non-trivial columns}\label{subsec:correctness_non-trivial-columns} % TODO: Writing
Experiment Setup

Result

Conclusion

\subsection{Summarized Results}\label{subsec:correctness_conclusions} % TODO: Writing

\subsection{Examine columns which led to false guesses}\label{subsec:correctness_examine-false-guesses} % TODO: Writing
