\chapter{Motivation}
Tables are an increasingly important way to store and organize data. Be it SQL databases, No-SQL databases or simple Excel tables, the ability to organize data and present it concisely is at the core of many projects.

In the best case, the structure of the data set has already been considered before creating them in order to clarify which data type can occur in a column, whether values can be left empty and which columns act as primary keys.

However, there are cases where this did not happen for various reasons; either because the data had to be saved quickly and there was no capacity for reasonable planning, or because the data is no longer available in the original format. % TODO: maybe a citation?

In this case, it is an important task to recover the missing metadata as accurately as possible. While information such as the data type and the existence of empty values is comparatively easy to find within a linear runtime, identifying primary keys is more difficult.

One challenge is to distinguish between primary key candidates, which are characterized by the fact that they do not contain duplicates, and practically usable primary keys. % TODO: write better
For example, even if a column containing comments from users does not contain duplicates, text that is on average 100 characters long is not really suitable as a key.

Another problem is the runtime needed to determine if a column contains duplicates. For very large tables with several million rows, this step can take some time. % TODO: how much time? cite related work
